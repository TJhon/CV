%!TEX TS-program = xelatex
%!TEX encoding = UTF-8 Unicode
% Awesome CV LaTeX Template for CV/Resume
%
% This template has been downloaded from:
% https://github.com/posquit0/Awesome-CV
%
% Author:
% Claud D. Park <posquit0.bj@gmail.com>
% http://www.posquit0.com
%
%
% Adapted to be an Rmarkdown template by Mitchell O'Hara-Wild
% 23 November 2018
%
% Template license:
% CC BY-SA 4.0 (https://creativecommons.org/licenses/by-sa/4.0/)
%
%-------------------------------------------------------------------------------
% CONFIGURATIONS
%-------------------------------------------------------------------------------
% A4 paper size by default, use 'letterpaper' for US letter
\documentclass[11pt,a4paper,]{awesome-cv}

% Configure page margins with geometry
\usepackage{geometry}
\geometry{left=1.4cm, top=.8cm, right=1.4cm, bottom=1.8cm, footskip=.5cm}


% Specify the location of the included fonts
\fontdir[fonts/]

% Color for highlights
% Awesome Colors: awesome-emerald, awesome-skyblue, awesome-red, awesome-pink, awesome-orange
%                 awesome-nephritis, awesome-concrete, awesome-darknight

\definecolor{awesome}{HTML}{11577c}

% Colors for text
% Uncomment if you would like to specify your own color
% \definecolor{darktext}{HTML}{414141}
% \definecolor{text}{HTML}{333333}
% \definecolor{graytext}{HTML}{5D5D5D}
% \definecolor{lighttext}{HTML}{999999}

% Set false if you don't want to highlight section with awesome color
\setbool{acvSectionColorHighlight}{true}

% If you would like to change the social information separator from a pipe (|) to something else
\renewcommand{\acvHeaderSocialSep}{\quad\textbar\quad}

\def\endfirstpage{\newpage}

%-------------------------------------------------------------------------------
%	PERSONAL INFORMATION
%	Comment any of the lines below if they are not required
%-------------------------------------------------------------------------------
% Available options: circle|rectangle,edge/noedge,left/right

\name{Jhon K. Flores Rojas}{}

\position{Data Scientist \textbar{} Data Analyst}
\address{Junín - Perú}

\mobile{+51 947178779}
\email{\href{mailto:fr.jhonk@gmail.com}{\nolinkurl{fr.jhonk@gmail.com}}}
\homepage{jhonkevinfr.netlify.app}
\github{tjhon}
\linkedin{jhonk-fr}
\twitter{JhonKevinFlore1}

% \gitlab{gitlab-id}
% \stackoverflow{SO-id}{SO-name}
% \skype{skype-id}
% \reddit{reddit-id}


\usepackage{booktabs}

\providecommand{\tightlist}{%
	\setlength{\itemsep}{0pt}\setlength{\parskip}{0pt}}

%------------------------------------------------------------------------------



% Pandoc CSL macros

\begin{document}

% Print the header with above personal informations
% Give optional argument to change alignment(C: center, L: left, R: right)
\makecvheader

% Print the footer with 3 arguments(<left>, <center>, <right>)
% Leave any of these blank if they are not needed
% 2019-02-14 Chris Umphlett - add flexibility to the document name in footer, rather than have it be static Curriculum Vitae
\makecvfooter
  {April 2024}
    {Jhon K. Flores Rojas~~~·~~~Curriculum Vitae}
  {\thepage}


%-------------------------------------------------------------------------------
%	CV/RESUME CONTENT
%	Each section is imported separately, open each file in turn to modify content
%------------------------------------------------------------------------------



\hypertarget{algunas-cosas-sobre-mi}{%
\section{Algunas cosas sobre mi}\label{algunas-cosas-sobre-mi}}

\begin{cvparagraph}
Poseo experiencia en diversos lenguajes de programación como Python, Julia y R. Además, cuento con habilidades para llevar a cabo análisis de datos, econometría, aprendizaje automático y evaluaciones de  impacto. Apliqué las destrezas en inferencia causal y tambien tengo conocimiento del uso de Large Language Models (LLMs).

También tengo competencia en la automatización y creación de librerías en estos lenguajes, lo que me permite contribuir eficazmente en la automatización de procesos. Además, he adquirido experiencia en colaboración en equipo a través de herramientas como GitHub en el ámbito de la programación.
\end{cvparagraph}

\hypertarget{educaciuxf3n}{%
\section{Educación}\label{educaciuxf3n}}

\begin{cventries}
    \cventry{Economía}{Universidad Nacional del Centro del Peru}{Huancayo, Peru}{2015 - 2019}{\begin{cvitems}
\item Economía
\end{cvitems}}
    \cventry{Becado}{OSINERGMIN}{Lima, Peru}{2020, enero - 2020, marzo}{\begin{cvitems}
\item Curso de Extensión Universitaria XVII
\end{cvitems}}
    \cventry{Becado}{OSIPTEL}{Lima, Peru}{2021, enero - 2021, marzo}{\begin{cvitems}
\item Curso de Extensión Universitaria XXV
\end{cvitems}}
\end{cventries}

\hypertarget{experiencia-laboral}{%
\section{Experiencia laboral}\label{experiencia-laboral}}

\begin{cventries}
    \cventry{Asistente de investigación}{D2CML}{Lima, Peru}{2022-May - Now}{\begin{cvitems}
\item Adaptación y traslado de métodos de estimación y gráficos de la "World Bank - R Econ Visual Library" (Causal Machine Learning e Inferencia Causal) a un entorno visual en Python.
\item Aplicación de Large Language Models (LLM) para la automatización y la clasificación de lenguaje natural.
\item Apoyo en la creación de la libreria de CSDID orientada a la inferencia causal con bigdata usando pyspark.
\item Apoyo en la creación de la libreria de HDMpy (Python) orientada en la inferencia causal
\item Automatización de procesos de compilación de libros utilizando flujos de trabajo en GitHub.
\item Contribución al desarrollo y documentación de bibliotecas como HDMjl en Julia, específicamente en su aplicación para la validación de modelos.
\item Corrección e implementación de nuevos métodos de los libros "Inference on Causal and Structural Parameters using ML and AI with R, Python, and Julia", utilizados en el curso 14.38 en MIT, y "Machine Learning and Causal Inference using Python", usados en el curso MGTECON-634 en Stanford.
\item Desarrollo de aplicaciones web utilizando Streamlit en Python.
\item Recolección y Tratamiento de Datos Espaciales de Pakistan
\item Recolección, Tratamiento, Manipulacion de Datos Espaciales de Etiopía
\item Soporte en la depuración y organización de datos para análisis encomendados por el investigador principal.
\item Tratamiento de datos del Ministerio del Interior -. Delitos
\end{cvitems}}
    \cventry{Asistente}{Universidad Nacional del Centro del Perú}{Huancayo, Peru}{2019}{\begin{cvitems}
\item Apoyo en la implementación del Sistema de Gestión de Calidad (SUNEDU) de la Facultad de Economía.
\item Generación y automatización de informes.
\end{cvitems}}
    \cventry{Practicas}{Rileco SAC}{Huancayo, Peru}{2020}{\begin{cvitems}
\item Apoyo en la recolección, limpieza y modelado de modelos econométricos.
\end{cvitems}}
\end{cventries}

\newpage

\hypertarget{habilidades-relevantes}{%
\section{Habilidades Relevantes}\label{habilidades-relevantes}}

\begin{cvskills}  
\cvskill 
    {Programación}
    {R, Python, SQL, Julia, Stata, Java} 
\cvskill 
    {Análisis de Datos}
    {Visualización, Limpieza de Datos, Pronósticos, Modelado, Comunicación, Inferencia, Aprendizaje Automático, Web Scraping} 
\cvskill 
    {Software}
    {Rstudio, PowerBi, Tableau, VSCode, NeoVim, Linux, Terminal, MSOffice} 
\cvskill 
    {Programación Web}
    {HTML, CSS, JavaScript, ReactJS} 
\cvskill 
    {Desarrollo de Paquetes}
    {R, Python, Julia} 
\cvskill 
    {Otros}
    {AWS (EC2), AZURE (Form Recognizer API), Git, Markdown, Quarto, Jupyter, SupaBase, Automatización} 
    
\end{cvskills}

\hypertarget{desarrollo-de-paquetes-y-librerias}{%
\section{Desarrollo de Paquetes y
Librerias}\label{desarrollo-de-paquetes-y-librerias}}

\hypertarget{personales}{%
\subsection{Personales}\label{personales}}

\small

\begin{itemize}
\tightlist
\item
  \href{https://github.com/TJhon/PeruData}{PeruData}
\item
  \href{https://github.com/TJhon/ssynthdid}{Ssynthdid} \normalsize
\end{itemize}

\hypertarget{como-asistente-de-investigaciuxf3n}{%
\subsection{Como Asistente de
Investigación}\label{como-asistente-de-investigaciuxf3n}}

\small

\begin{itemize}
\tightlist
\item
  \href{https://github.com/d2cml-ai/python_visual_library}{Python Visual
  Library}
\item
  \href{https://github.com/d2cml-ai/HDMjl.jl}{High Dimensional Methods -
  Julia}
\item
  \href{https://github.com/d2cml-ai/Synthdid.jl}{Synthdid - Julia}
\item
  \href{https://github.com/d2cml-ai/synthdid.py}{Synthdid - Python}
\item
  \href{https://github.com/d2cml-ai/osrmareas}{OSMR Areas - R}
\item
  \href{https://github.com/alexanderquispe/osrm_python}{OSMR Python -
  Python}
\item
  \href{https://github.com/d2cml-ai/csdid}{CSDID - Python} \normalsize
\end{itemize}

\hypertarget{participaciones-en-concursos-de-machine-learning}{%
\section{Participaciones en Concursos de Machine
Learning}\label{participaciones-en-concursos-de-machine-learning}}

\begin{cvhonors}
    \cvhonor{}{Maratona Behind the Code}{IBM - Latinoamerica}{2021}
    \cvhonor{}{Datathon2021 - Entel}{Entel}{2021}
    \cvhonor{}{Datathon Interbank}{Interbank}{2020}
\end{cvhonors}



\end{document}
